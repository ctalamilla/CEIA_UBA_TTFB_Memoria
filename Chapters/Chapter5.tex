% Chapter Template

\chapter{Conclusiones} % Main chapter title

\label{Chapter5} % Change X to a consecutive number; for referencing this chapter elsewhere, use \ref{ChapterX}
Este capítulo presenta las conclusiones generales del trabajo, los aportes más relevantes obtenidos y describe el grado de cumplimiento de los objetivos planteados. Asimismo, se incluyen observaciones sobre la planificación y los riesgos, y se señalan posibles líneas de desarrollo futuro. 

%----------------------------------------------------------------------------------------
%	SECTION 1
%----------------------------------------------------------------------------------------

\section{Conclusiones generales }
El trabajo permitió caracterizar la dinámica temporal de la superficie de agua en el Salar de Llullaillaco y su relación con los principales índices climáticos asociados al ENSO. A partir de la construcción de un cubo temporal mensual continuo y de la aplicación de metodologías estadísticas y de aprendizaje profundo, se alcanzaron hallazgos relevantes sobre la influencia climática regional en la hidrología altoandina. 

Entre los principales resultados se destaca que el SOI emergió como el índice con mayor capacidad explicativa sobre la superficie de agua, confirmado por los modelos VAR y la prueba de causalidad de Granger. La integración de SARIMA y VAR permitió capturar la estacionalidad y las interdependencias multivariadas, mientras que la segmentación automática con U-Net facilitó la generación de series homogéneas sin necesidad de etiquetado manual. En síntesis, se validó la hipótesis de que la variabilidad climática global, en particular el ENSO, constituye un factor determinante en la dinámica hídrica del salar.


A partir de la planificación inicial y las pruebas desarrolladas, se alcanzaron los siguientes logros y hallazgos principales:

\begin{itemize}
    \item Se construyó un cubo temporal mensual continuo de la superficie de agua (1984--2022), combinando interpolación e imputación estacional para cubrir vacíos y asegurar homogeneidad en la serie.
    \item La segmentación con U-Net, entrenada mediante el esquema \emph{knowledge distillation}, permitió generar máscaras mensuales de agua de manera automática, evitando la necesidad de etiquetado manual y aportando consistencia espacial y temporal.
    \item A partir de estas máscaras se derivó la serie de Área, que mostró estacionariedad solo tras diferenciación, mientras que las series climáticas ONI, SOI y MEI fueron estacionarias desde el inicio.
    \item Los modelos SARIMA mostraron buen ajuste, destacándose la serie diferenciada de Área con residuos consistentes con ruido blanco; el SOI resultó más complejo de modelar por su alta variabilidad.
    \item El modelo VAR(3) capturó interdependencias multivariadas entre ONI, SOI y MEI, confirmando la influencia significativa del SOI sobre el Área, resultado reforzado por la prueba de causalidad de Granger y las respuestas al impulso.
    \item Se realizaron pronósticos mediante \emph{rolling forecast} con SARIMA y VAR, validando la capacidad predictiva de ambas metodologías y evidenciando limitaciones en la captura de picos extremos.
    \item Se cumplieron la mayoría de los requisitos planteados en la planificación: construcción de series homogéneas, modelado SARIMA/VAR, segmentación automática con U-Net, validación de resultados y visualización. Permanecen en progreso las métricas de precisión categórica y las pruebas cruzadas con Sentinel-2.
    \item Los riesgos identificados (brechas satelitales, errores en índices) no se materializaron de manera crítica; las medidas de mitigación fueron efectivas, aunque la alta variabilidad del SOI y la necesidad de preprocesamiento riguroso se mantuvieron como los principales desafíos.
\end{itemize}


El trabajo permitió validar la hipótesis inicial: la variabilidad climática global asociada al ENSO, en especial la expresada por el SOI, constituye un factor determinante en la dinámica hídrica del Salar de Llullaillaco.

%----------------------------------------------------------------------------------------
%	SECTION 2
%----------------------------------------------------------------------------------------

\section{Próximos pasos}

El trabajo abre la posibilidad de ampliar la metodología hacia nuevas aplicaciones y mejoras técnicas. Entre ellas, la extensión a otros salares de la Puna y el altiplano para evaluar la consistencia regional, la incorporación de sensores adicionales como Sentinel-2, y la exploración de modelos de inteligencia artificial avanzados (LSTM, transformers) capaces de capturar relaciones no lineales y patrones complejos. 

Asimismo, resulta prioritario profundizar el análisis de incertidumbre mediante modelos que contemplen heterocedasticidad y distribuciones no gaussianas, e integrar variables hidrometeorológicas locales (precipitación, temperatura, evapotranspiración) que complementen los indicadores climáticos globales. Estos pasos permitirán consolidar la base metodológica y avanzar hacia sistemas predictivos robustos que respalden la gestión ambiental en ecosistemas de altura.
